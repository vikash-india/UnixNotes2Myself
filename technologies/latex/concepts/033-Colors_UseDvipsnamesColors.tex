% Description: Using Colors in Latex

% Note
% 1. The colors in Latex are supported through color and xcolor packages.
% 2. The colour system provided by the packages color and xcolor is built around the idea of colour models, the colour
%    mode and the colour names.
% 3. The colors supported by a driver may vary.
% 4. The model based on colour names is very intuitive, even though the list of available names is limited.

\documentclass{article}

% Added for colors
% 1. The usenames makes the names in the corresponding driver name model available.
% 2. This option can be omitted in xcolor.
\usepackage[usenames, dvipsnames]{color}

\begin{document}

\section{Colors}
Use \texttt{ccolor} package to change the colour of elements in \LaTeX.

\subsection{All Text in Color}
{\color{ForestGreen}
Lorem ipsum dolor sit amet, consectetur adipiscing elit. Fusce sed eros at leo congue consequat quis quis erat. Mauris
sit amet urna ac nulla maximus sodales.}

\subsection{Selected Text in Color}
% The syntax is \textcolor{colorname}{text}.
The selected \textcolor{red}{text} can also be displayed in color.

\subsection{Selected Text in Background Color}
% The syntax is \colorbox{BurntOrange}{this text}.
The background colour of some text can also be easily set. For instance, change to orange the background of
\colorbox{BurntOrange}{this text} and then continue typing.

\subsection{Line in Color}
{\color{RubineRed} \rule{\linewidth}{0.5mm} }

\end{document}